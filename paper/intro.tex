\section{Introduction}

Consider the following optimization problem, {\em capacitated $k$-means
  clustering}:
\begin{itemize}
\item {\bf input:} A set $\cal P$ of points (the
{\em clients}), a positive integer $k$, and a $k$-tuple
$(b_1, \ldots, b_k)$ of nonnegative integers whose sum is the number
of clients.
\item {\bf output:} A $k$-tuple $(c_1, \ldots, c_k)$ of points
(the {\em centers}) and an assignment of clients to centers, such that
center $c_i$ is assigned at most $b_i$ clients.
\item {\bf objective:} Minimize the sum of squared
distances from clients to their assigned centers.
\end{itemize}
We say the assignment {\em respects} the capacities to indicate that
the number of clients assigned each center is no more than the
corresponding capacity.

The problem is theoretically intractable so
we do not propose an algorithm that is guaranteed to find the best
solution.  Instead, we consider a natural local-search heuristic.
Before outlining this heuristic, we review a traditional clustering
problem and the traditional method for solving it.

\subsection{Traditional {\em $k$-means}}
There is a well-known clustering problem, often 
called {\em $k$-means clustering}, that does not involve capacities:
\begin{itemize}
\item {\bf input:} A set $\cal P$ of points (the
{\em clients}) and a positive integer $k$
\item {\bf output:} A $k$-tuple $(c_1, \ldots, c_k)$ of points
(the {\em centers}) and an assignment of clients to centers.
\item {\bf objective:} Minimize the sum of squared
distances from clients to their assigned centers.
\end{itemize}
This optimization problem is also\footnote{In fact, the intractability
  of this problem implies the intractability of the capacitated
  problem } theoretically intractable (it is NP-hard~\cite{***}) but 
it is often addressed in practice by an algorithm, sometimes
called the {\em $k$-means algorithm} and sometimes called {\em
  Lloyd's algorithm}.  The algorithm selects initial values for
the centers $c_1, \ldots, c_k$, and then repeats the following two
steps.
\begin{itemize}
\item {\bf Assignment step:} Assign each client to the closest center.
\item {\bf Update step:} For each center $c_i$, find the centroid of
  the clients assigned to it, and update $c_i$ to be this centroid.
\end{itemize}
This local-search algorithm continues until further repetition does
not change the locations of the centers.

Because the $k$-means clustering problem is considered theoretically
intractable, it is believed that no algorithm is guaranteed both to
terminate quickly and find an optimal solution.  The $k$-means
algorithm in fact fails in both respects: it sometimes requires a huge
amount of time (exceeding any fixed polynomial in $n$, the number of
clients) and when it terminates the output solution might not be the
best solution possible.  However, it tends to work well and is widely
used.

Note that in each of the two steps the algorithm makes choices so as
to minimize the sum of squared client-to-center distances.
\begin{itemize}
\item   In the
assignment step, each client's contribution to that sum is the
squared distance of that client to the center to which it is assigned,
so the algorithm chooses that center so as to minimize that
contribution.

\item In the update step, consider the set of clients assigned to some
  center $c_i$.  The point $\hat c$ that minimizes the sum of squared
  distances from those clients to $\hat c$ is the centroid of those
  points, so the algorithm updates $c_i$ to be that centroid.
\end{itemize}

\subsection{Capacitated {\em $k$-means}}

There is a natural way to adapt the $k$-means algorithm to address the
capacitated problem.  The assignment step should be replaced with one
that chooses the assignment of clients to centers to minimize the sum
of squared distances {\em subject to} the capacity constraints.  Thus
the modified algorithm at a high level is as follows.

The {\em capacitated $k$-means algorithm} selects initial values for
the centers $c_1, \ldots, c_k$, and then repeats the following two
steps.
\begin{itemize}
\item {\bf Assignment step:} Choose the assignment of clients to
  centers that minimizes the sum of squared client-to-assigned-centers
  distances over all assignments that respect the capacities.
\item {\bf Update step:} For each center $c_i$, find the centroid of
  the clients assigned to it, and update $c_i$ to be this centroid.
\end{itemize}
The algorithm continues until further repetition does not change the
locations of the centers.

Of course, like the $k$-means algorithm, the capacitated $k$-means
algorithm is guaranteed neither to terminate quickly nor to output an
optimal solution.  However, note that it does output a solution that
respects the capacities.  

The new assignment step is considerably more complicated than merely
assigning each client to the nearest center.  However, it can be
carried out in polynomial time.  The assignment step can be formulated
as an instance of one classical problem called the {\em assignment
  problem}.  Using, for example, the algorithm of~\cite{FredmanT}
solves the problem in $O(n^3)$ time.  ~\cite{GabowT1989}.
Our current implementation uses an implementation due to
Goldberg~\ref{Goldberg97}of an algorithm for minimum-cost flow.

In previous work, XXXXX proposed an algorithm equivalent to the
capacitated $k$-means algorithm, except that a local-search heuristic
was proposed to carry out the assignment step; use of such a heuristic
does not guarantee that the step will be completed in polynomial time
and does not guarantee that the solution obtained in that step is the
best assignment.

\subsection{Voronoi diagrams and power diagrams}

Fix a set of centers in a metric space.  For each center $c$, the
corresponding {\em Voronoi cell} is the set of points of the metric
space that are no farther from $c$ than from any other center.
The {\em Voronoi diagram} of the centers is the set of Voronoi cells.
When the metric space is Euclidean (e.g. the plane $\R^2$), each
Voronoi cell is a convex polyhedron.  

A {\em power diagram} is a generalization of a Voronoi diagram.  With
each center $c$ is associated a real number $w_c$, called the {\em weight} of
the center.  The weighted squared distance of a point $p$ to a cell
$c$ is the squared distance from $p$ to $c$ minus the weight of $c$.  
 The {\em power cell} of $c$ consists of all points $p$ whose weighted squared
 distance to $c$ is no more than the weighted squared distance to any
 other center.

 Consider the case in which the metric space is the Euclidean plane
 $\R^2$.  Subtracting the same number from all weights does not change
 the cells.  Thus we can assume without loss of generality that all
 weights $w_c$ are negative.  Now consider each point $(x,y)$ in the
 metric space $\R^2$ as a point $(x,y,0)$ in $\R^3$.  For each center
 $c=(x,y)$, define the {\em modified center} $\hat c$ to be the point
 $(x,y, \sqrt{-w_c})$.  As a consequence, for any point $p=(x,y)$ in
 $\R^2$, the squared weighted distance from $p$ to a center $c$ equals
 the squared distance from $(x,y,0)$ to the corresponding modified
 center $\hat c$.  This in turn means that, for each center $c$, the
 power cell of $c$ in $\R^2$ is exactly the intersection with the
 plane $\set{(x,y,z)\ :\ z=0}$ of the Voronoi cell of $\hat c$ in
 $\R^3$.  Because the Voronoi cells are convex polyhedra, so are the
 power cells.

\subsection{Representing the solution as a power diagram}

What do Voronoi diagrams and power diagrams have to do with $k$-means?
According to the assignment step for ordinary $k$-means, each client
is assigned to the nearest center; under these circumstances, the
client obviously belongs to the Voronoi cell of that client.  

According to the assignment step for capacitated $k$-means,
clients are assigned subject to capacity constraints, so might not be
assigned to the nearest centers, so might not belong to the Voronoi
cells of their assigned centers.  However, it has been shown that
under these circumstances there are weights for which each client
belongs to the power cell of its assigned center.

\subsection{The result of the capacitated $k$-means algorithm}

It follows that the solution obtained by the capacitated $k$-means
algorithm is guaranteed to have the following properties:
\begin{enumerate}
\item Each client is assigned to a center.
\item The number of clients assigned to each center equals the
  capacity of the center.
\item Each center is at the centroid of the clients assigned to it.
\item The plane can be decomposed into $k$ convex polygons, such that
  each polygon contains the clients assigned to one of the centers.
\end{enumerate}
We cannot prove that the polygons are well-rounded; in principle, some
might be long and skinny.  However, because the $k$-means algorithm is
known to be an effective heuristic for minimizing the sum of squared
distances from clients to centers, there is reason to believe that the
capacitated $k$-means algorithm is similarly an effective heuristic
for minimizing the sum of squared client-to-center distances subject
to capacities.  We therefore expect that typically the polygons will
have small aspect ratio (ratio of diameter to width).

We will later give empirical evidence in support of that observation.

