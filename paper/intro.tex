\section{Introduction}

Consider the following optimization problem, {\em capacitated $k$-means
  clustering}:
\begin{itemize}
\item {\bf input:} A set $\cal P$ of points (the
{\em clients}), a positive integer $k$, and a $k$-tuple
$(b_1, \ldots, b_k)$ of nonnegative integers whose sum is the number
of clients.
\item {\bf output:} A $k$-tuple $(c_1, \ldots, c_k)$ of points
(the {\em centers}) and an assignment of clients to centers, such that
center $c_i$ is assigned at most $b_i$ clients.
\item {\bf objective:} Minimize the sum of squared
distances from clients to their assigned centers.
\end{itemize}
We say the assignment {\em respects} the capacities to indicate that
the number of clients assigned each center is no more than the
corresponding capacity.

The problem is theoretically intractable so
we do not propose an algorithm that is guaranteed to find the best
solution.  Instead, we consider a natural local-search heuristic.
Before outlining this heuristic, we review a traditional clustering
problem and the traditional method for solving it.

\subsection{Traditional {\em $k$-means}}
There is a well-known clustering problem, often 
called {\em $k$-means clustering}, that does not involve capacities:
\begin{itemize}
\item {\bf input:} A set $\cal P$ of points (the
{\em clients}) and a positive integer $k$
\item {\bf output:} A $k$-tuple $(c_1, \ldots, c_k)$ of points
(the {\em centers}) and an assignment of clients to centers.
\item {\bf objective:} Minimize the sum of squared
distances from clients to their assigned centers.
\end{itemize}
This optimization problem is also\footnote{In fact, the intractability
  of this problem implies the intractability of the capacitated
  problem } theoretically intractable (it is NP-hard~\cite{***}) but 
it is often addressed in practice by an algorithm, sometimes
called the {\em $k$-means algorithm} and sometimes called {\em
  Lloyd's algorithm}.  The algorithm selects initial values for
the centers $c_1, \ldots, c_k$, and then repeats the following two
steps.
\begin{itemize}
\item {\bf Assignment step:} Assign each client to the closest center.
\item {\bf Update step:} For each center $c_i$, find the centroid of
  the clients assigned to it, and update $c_i$ to be this centroid.
\end{itemize}
This local-search algorithm continues until further repetition does
not change the locations of the centers.

Because the $k$-means clustering problem is considered theoretically
intractable, it is believed that no algorithm is guaranteed both to
terminate quickly and find an optimal solution.  The $k$-means
algorithm in fact fails in both respects: it sometimes requires a huge
amount of time (exceeding any fixed polynomial in $n$, the number of
clients) and when it terminates the output solution might not be the
best solution possible.  However, it tends to work well and is widely
used.

Note that in each of the two steps the algorithm makes choices so as
to minimize the sum of squared client-to-center distances.
\begin{itemize}
\item   In the
assignment step, each client's contribution to that sum is the
squared distance of that client to the center to which it is assigned,
so the algorithm chooses that center so as to minimize that
contribution.

\item In the update step, consider the set of clients assigned to some
  center $c_i$.  The point $\hat c$ that minimizes the sum of squared
  distances from those clients to $\hat c$ is the centroid of those
  points, so the algorithm updates $c_i$ to be that centroid.
\end{itemize}

\subsection{Capacitated {\em $k$-means}}

There is a natural way to adapt the $k$-means algorithm to address the
capacitated problem.  The assignment step should be replaced with one
that chooses the assignment of clients to centers to minimize the sum
of squared distances {\em subject to} the capacity constraints.  Thus
the modified algorithm at a high level is as follows.

The {\em capacitated $k$-means algorithm} selects initial values for
the centers $c_1, \ldots, c_k$, and then repeats the following two
steps.
\begin{itemize}
\item {\bf Assignment step:} Choose the assignment of clients to
  centers that minimizes the sum of squared client-to-assigned-centers
  distances over all assignments that respect the capacities.
\item {\bf Update step:} For each center $c_i$, find the centroid of
  the clients assigned to it, and update $c_i$ to be this centroid.
\end{itemize}
The algorithm continues until further repetition does not change the
locations of the centers.

Of course, like the $k$-means algorithm, the capacitated $k$-means
algorithm is guaranteed neither to terminate quickly nor to output an
optimal solution.  However, note that it does output a solution that
respects the capacities.  

The new assignment step is considerably more complicated than
